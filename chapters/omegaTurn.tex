%!TEX root = /Users/andy/Documents/Academics/Dissertation/thesis.tex

\begin{comment}
	Points to make here:
	Navigatoin is worthy of study
	Omega turn is a critical part of navigation
	
\end{comment}

\chapter{Neural activity of the Omega Turn}

\section{Background}

\newthought{Navigation is a goal directed locomotion}, common across species, in which an organism must generate asymmetry in their motor program to steer toward or away from a cue.  At the neuronal level, navigation requires the temporal coordination of different motor programs.  How does an animal generate asymmetry in a locomotion program allowing it to steer and change direction?  \textit{C. elegans} navigation is regulated by ventrally or dorsally biased head swings (Iino and Yoshida, 2009) and deep ventral omega turns that reorient the worm in the opposite direction.  The worm's escape response consists of this omega turn. The neural circuit for the \textit{C. elegans} escape response employs mechanosensory neurons, command neurons and motor neurons to govern decision making, the coordination of motor programs and turning behavior. The escape response in \textit{C. elegans} provides a platform upon which to investigate how an animal's nervous system carries out a change in direction.  




The escape response in \textit{C. elegans} can be elicited by gently touching the anterior of the worm. The response is very stereotyped. The worm first ceases its forward locomotion and exploratory head movements (Alkema et al., 2005), it then moves backward away from the stimulus (Chalfie et al., 1985), comes to a stop, and then bends its head ventrally and  exhibits a deep ventral turn before finally reinitiating forward locomotion.  The entire sequence is commonly referred to as an omega turn. 

The analysis of this stereotyped omega turn provides an opportunity to identify the molecular and neuronal mechanisms  the nervous system employs to translate sensory information into navigational behavior.  


In this chapter we present ongoing work to explore the neural activity underlying the omega turn. A real-time tracking system was developed to record intracellular calcium transients in single neurons while simultaneously monitoring the macroscopic behavior of a freely moving worm as it undergoes an escape response.

[CHECK THE RIM DATA] This work lays to rest some confusion in the field about the contribution of individual neurons to the escape response.  [TRUE?]

 

\section{Dual-Magnification Calcium and Behavior Imaging}
To study the neural activity of an omega turn we required a system to simultaneously observe neural activity in a single neuron and observe the worm's behavior. 

built a microscope to simultaneously track a single target on a worm, record calcium transients and record macroscopic behavior. To do this 

similar in principal to prior work [schaeffer, shawn lockery], but superior on a number of basis. First, it is the first worm tracker, to our knowledge, that uses a  spinning disk confocal microscope which provides crisp backgrounds. Second, it uses inexpensive USB camera to do the tracking, but leverages the real-time computer vision software from the CoLBeRT system to do very rapid feedback. Our feedback operates at 30 Hz, with latencies that are evidently low enough to sufficently track features on the worm, even when it reverses suddenly. 










C. elegans crawls on its side in a sinusoidal pattern by propagating dorso-ventral flexures along the anterior-posterior axis of the body.  


 The escape response is initiated by mechanosensory neurons which transduce sensory information through locomotion command neurons to excitatory and inhibitory motor neurons that innervate the body wall muscles (Chalfie et al., 1985)



Gentle touch to the anterior of the worm induces an escape response where the animal stops exploratory head movements (Alkema et al., 2005) and moves backward away from the stimulus (Chalfie et al., 1985).  The escape response is initiated by mechanosensory neurons which transduce sensory information through locomotion command neurons to excitatory and inhibitory motor neurons that innervate the body wall muscles (Chalfie et al., 1985).  Reinitiation of forward locomotion is accompanied by a steep ventral head bend followed by a deep turn (omega turn) that redirects the worm away from the stimulus.  

The analysis of the worm escape response gives us the  opportunity to identify the molecular and neuronal mechanisms  the nervous system employs to translate sensory information into navigational behavior. 

Genetically encoded calcium indicators in combination with  computer vision worm tracking will be used to determine the activity patterns of mechanosensory neurons, locomotion command neurons, neck motor neurons and neck muscles during turning behavior.  

Unraveling the neural circuitry required for turning behavior will lead to the understanding of the neural and molecular machinery that confers flexibility in the output of coordinated motor programs.



\subsection{Methods}


\subsubsection{Optogenetic Stimulation Apparatus}



\subsubsection{Optogenetic Stimulation in Moving Worm}


\subsubsection{Calcium Imaging Apparatus}
\subsubsection{Calcium Imaging in Moving Worm}

\section{Manuscript Information}
The work presented here was done by Andrew M. Leifer in close collaboration with Christopher Clark and Mark Alkema of the Alkema Lab. Andrew Leifer built the hardware with Mason Klein. Christopher Clark and Andrew Leifer together performed all of the experiments. Christopher generated the worm strains. Andrew wrote all software and analyzed the data. Both Andrew and Christopher wrote the manuscript.






Animals navigate their environment to locate favorable conditions, find food, find a mate and avoid predation. While principles of basic locomotion for many animals including humans are well understood, little is known about the molecular and neural mechanisms of goal directed locomotion.  Navigation requires the integration of various sensory stimuli followed by precise orchestration of musculature to produce goal directed movement.  Animals ranging from fruit flies to humans integrate visual, olfactory, and mechanical sensory inputs to determine their current and desired locations.  How does an animal process sensory information to navigate its environment?  Defining sensorimotor circuits requires not only a detailed knowledge of the neural connectivity of the nervous system, but also the ability to manipulate the functions of the component neurons. The simplicity and completely defined synaptic connectivity of the nervous system of the nematode C. elegans provides a unique opportunity to fully define the cellular and molecular pathways which regulate navigational behavior.

Navigation is a goal directed locomotion in which an animal must generate asymmetry in their locomotion program to steer toward or away from a cue.  At the neuronal level, navigation requires the temporal coordination of different motor programs.  How does an animal generate asymmetry in a locomotion program allowing it to steer and change direction?  C. elegans navigation is regulated by ventrally or dorsally biased head swings (Iino and Yoshida, 2009) and deep ventral omega turns that reorient the worm in the opposite direction.  The neural circuit for the C. elegans escape response provides a paradigm to investigate how an animal changes direction.  This circuit employs mechanosensory neurons, command neurons and motor neurons to govern decision making, the coordination of motor programs and turning behavior. C. elegans crawls on its side in a sinusoidal pattern by propagating dorso-ventral flexures along the anterior-posterior axis of the body.  Gentle touch to the anterior of the worm induces an escape response where the animal stops exploratory head movements (Alkema et al., 2005) and moves backward away from the stimulus (Chalfie et al., 1985).  The escape response is initiated by mechanosensory neurons which transduce sensory information through locomotion command neurons to excitatory and inhibitory motor neurons that innervate the body wall muscles (Chalfie et al., 1985).  Reinitiation of forward locomotion is accompanied by a steep ventral head bend followed by a deep turn (omega turn) that redirects the worm away from the stimulus.  The analysis of the worm escape response gives us the unique opportunity to identify the molecular and neuronal mechanisms of how the nervous system translates sensory information into navigational behavior. Genetically encoded calcium indicators in combination with machine-vision worm tracking will be used to determine the activity patterns of mechanosensory neurons, locomotion command neurons, neck motor neurons and neck muscles during turning behavior.  Unraveling the neural circuitry required for turning behavior will lead to the understanding of the neural and molecular machinery that confers flexibility in the output of coordinated motor programs.

I will express the fluorescent calcium sensor, GCaMP3 (Tian et al., 2009), in the AVM and ALM mechanosensory neurons, the AVA and AVB locomotion command neurons; the RMD, SMD and RIV neck motor neurons; and muscles with the mec-4, rig-3, lgc-55, lim-4 and myo-3 promoters (Sagasti et al., 1999; Pirri et al., 2009; Brockie et al., 2001; Ardizzi and Epstein, 1987).  With these transgenic lines, I will analyze mechanosensory neuron, command neuron, motor neuron and neck muscle flourescence to determine their activity during turning behavior.

\begin{comment}
Alkema, M.J., Hunter-Ensor M., Ringstad N. and Horvitz H.R. (2005). Tyramine functions independently of octopamine in the Caenorhabditis elegans nervous system. Neuron 46, 247- 260.

Ardizzi J.P. and Epstein, H.E. (1987). Immunochemical Localization of Myosin Heavy Chain Isoforms and Paramyosin in Developmentally and Structurally Diverse Muscle Cell Types of the Nematode Caenorhabditis elegans. Journal of Cell Biology 105, 2763-2770.

Brockie, P.J., Madsen, D.M., Zheng, Y., Mellem, J. and Maricq A.V. (2001). Differential expression of glutamate receptor subunits in the nervous system of Caenorhabditis elegans and their regulation by the homeodomain protein UNC-42. Journal of Neuroscience 21, 1510-1522.

Chalfie, M., Sulston, J.E., White, J.G., Southgate, E., Thomson, J.N. and Brenner, S. (1985). The Neural Circuit for Touch Sensitivity in Caenorhabditis elegans. Journal of Neuroscience 5, 956-964.

Gray, J.M., Hill, J.J. and Bargmann C.I. (2005). A circuit for navigation in Caenorhabditis elegans. Proceedings of the National Academy of Sciences of the USA 102, 3184-3191.

Iino, Y. and Yoshida, K. (2009). Parallel use of two behavioral mechanisms for chemotaxis in Caenorhabditis elegans. Journal of Neuroscience 29, 5370-5380.

Pirri, J.K., McPherson, A.D., Donnelly, J.L., Francis, M.M. and Alkema, M.J. (2009) A tyramine-gated chloride channel coordinates distinct motor programs of a Caenorhabditis elegans escape
response. Neuron 62, 526-538.

Sagasti, A., Hobert, O., Troemel, E.R., Ruvkun, G. and Bargmann, C.I. (1999). Alternative olfactory neuron fates are specified by the LIM homeobox gene lim-4. Genes \&  Development. 13, 1794- 1806.

Tian, L., Hires, S.A., Mao, T., Huber, D., Chiappe, M.E., Chalasani, S.H., Petreanu, L., Akerboom, J., McKinney, S.A., Schreiter, E.R., Bargmann, C.I., Jayaraman, V., Svoboda, K. and Looger, L.L. (2009). Imaging neural activity in worms, flies and mice with improved GCaMP calcium indicators. Nature Methods 6, 875-881.
\end{comment}
