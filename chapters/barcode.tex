%!TEX root = /Users/andy/Documents/Academics/Dissertation/thesis.tex




multiplexed snp genotyping site:ncbi.nlm.nih.gov

\chapter{Sub-micrometer Geometrically Encoded Fluorescent Barcodes Self-Assembled from DNA}

\section{Introduction}
%Comment to chenxiang: found the beginning cheesy,modified it a bit
\newthought{In biology and medicine} researchers often use fluorescence microscopy to visualize nanometer to micrometer-sized entities. In many cases it is desirable to visualize more than one class of objects simultaneously and unambiguously. As such there is a need to develop suitable 
fluorescent tags (barcodes) for multiplexed imaging applications. Most previously 
described fluorescent barcodes are constructed using either intensity encoding \citep{han_quantum-dot-tagged_2001,xu_multiplexed_2003,li_multiplexed_2005,livet_transgenic_2007,fournier-bidoz_facile_2008,lin_self-assembled_2007,marcon_--fly_2010} or 
geometrical encoding \citep{nicewarner-pena_submicrometer_2001,gudiksen_growth_2002,braeckmans_encoding_2003,dejneka_rare_2003,geiss_direct_2008,pregibon_multifunctional_2007,xiao_direct_2009,li_controlled_2010}. Intensity encoding relies on the combination of multiple 
spectrally differentiable fluorophores in a controlled molar ratio. Geometrical encoding, 
on the other hand, is obtained by separating optical features beyond the microscope’s 
resolution limit (typically \textasciitilde250 nm for diffraction-limited imaging and \textasciitilde 50 nm for 
current super-resolution imaging) and arranging them in a specific geometric pattern. The 
successful construction and decoding of intensity-encoded barcodes depends heavily on 
the fluorescent labeling efficiency and the imaging instrument’s ability to precisely detect 
different fluorescent intensity levels. Such requirements may pose practical difficulties 
for achieving robust encoding and limit the multiplexing power of the intensity-encoded 
barcodes. In comparison, geometrically encoded barcodes are more robust and may be 
constructed and detected even when the labeling efficiency and imaging conditions are 
%Comment to chenxiang: isn't his DNA barcode both geometrically encoded and intensity encoded?
not ideal. The multiplexing capability of geometrically encoded barcodes increases 
exponentially as additional spatially distinguishable fluorophores are incorporated. 
Therefore, larger barcode libraries may be more easily accessible through geometrical 
encoding, provided that a rigid structural scaffold capable of defining the spatial 
arrangement of the fluorescent molecules is available. To date, despite the remarkable 
success in synthesizing fluorescent barcodes for in vitro multiplexed detection, very little 
effort has been made to create robust single-molecule barcodes suitable as in situ imaging 
probes. In addition, most existing fluorescent barcodes range from 2 \textmu m to 100 \textmu m in 
size, leaving the construction of fluorescent barcodes with largest dimension less than 1 
\textmu m an underexplored research area (with only a few reports  \citep{li_multiplexed_2005,lin_self-assembled_2007,li_controlled_2010,levsky_single-cell_2002} and no more than 11 
distinct barcodes experimentally demonstrated). Here we report a group of geometrically 
encoded fluorescent barcodes self-assembled from DNA that can be used to tag yeast 
cells. These barcodes are 400–800 nm in length, structurally rigid, biocompatible, 
reprogrammable in a modular fashion and easy to decode using epi-fluorescence, total internal reflection fluorescence (TIRF) or 
super-resolution fluorescence microscopy. As evidence of the multiplexing power of the 
system, 216 distinct barcode species (20 times more than previously experimentally 
demonstrated systems) were constructed and resolved using diffraction-limited TIRF 
microscopy. 


Structural DNA nanotechnology takes advantage of the well-defined double 
helical structure of DNA and the highly predictable Watson-Crick base-paring rules to 
self-assemble designer nano-objects and devices. 17-22 In recent years, DNA origami has 
emerged as a prominent method to fabricate two- and three-dimensional structures with 
sizes of tens to hundreds of nanometers.23-30 By folding a long, single-stranded DNA 
molecule (a scaffold strand, often times an M13 viral genomic DNA or its derivatives) 
with the help of many short synthetic DNA oligonucleotides (staple strands), this 
approach generates complex, shape-controlled, fully addressable nanostructures. With 
certain functional groups attached to selected staple strands or their extensions, such 
nanostructures can be used to organize fluorescent guest molecules, including small 
organic molecules31-33 as well as metallic34 and semi-conductive nano-particles35. In 
addition, individual DNA-origami nanostructures can be joined together in a 
programmable way to make micrometer-sized structures while maintaining their unique 
nanometer scale spatial addressability36,37. These properties make DNA origami 
promising material to build robust fluorescent barcodes, as control over the exact ratio of 
different fluorophores allows intensity encoding while spatial positioning of fluorophores 
facilitates geometrical encoding and can help minimize undesired inter-fluorophore 
quenching. 






\section{Manuscript Information}
\subsection{Submitted for Publication As}
A version of this chapter has been submitted for publication in the journal \textit{Nature Nanotechnologies}.
% in \citep{leifer_optogenetic_2011}:
%\bibentry{leifer_optogenetic_2011}

\subsection{The Author's Contribution}
Andrew M.~Leifer conceived of and wrote the software to analyze TIRF images and identify and decode the barcodes. He wrote portions of the manuscript and generated one of the supplementary figures. 