%!TEX root = /Users/andy/Documents/Academics/Dissertation/thesis.tex

\chapter{Conclusion}
\lettrine{T}{he development of new tools and techniques} can drive scientific discovery when carefully applied. In this thesis I developed a new instrument to non-invasively manipulate neural activity with single-neuron specificity  in an intact freely moving worm. At the time of its creation, this was the first instrument of its kind. I also developed a second instrument to record calcium transients from a moving worm while simultaneously recording the worm's behavior. I used these tools to gain  a better understanding of how the nervous system of the nematode \textit{C. elegans} propagates bending waves and how it transitions from forward to reverse locomotion.

Both of the tools presented here  rely on advances in the fields of optogenetics and computer vision. In particular, the recent accessibility of powerful computer vision libraries  has wide applications to quantitative biology. In this thesis I also applied computer vision techniques towards the automatic identification of  DNA origami nano-barcodes. DNA nano-barcodes are a promising platform for developing next generation DNA microarray and on-chip tools.  The software I developed to automatically identify and decode barcodes provides a critical ingredient for future developments in this platform. 

Both  computer vision and optogenetics are technologies undergoing rapid development.
For computer vision, Moore's law has provided ever increasing computational power at decreasing cost. Computational power alone will immediately improve both the spatial and temporal resolution of the two \textit{C. elegans} instruments presented here. Similarly, recent trends like augmented reality for consumers, and faster data connectivity standards like Intel's Thunderbolt, will make the low-latency high frame-rate video acquisition  required for the work presented here more affordable and accessible.


In the realm of of optogenetics, the underlying technology is still nascent. The coming years will likely see the creation of a more diverse and robust optical toolkit that includes optogenetic proteins operating at new wavelengths, with greater sensitivity and with narrower spectra.  


Genetically encoded reporters for  neural activity are also undergoing rapid development. GCaMP5, a successor to the GCaMP3 calcium indicator used here, is rumored to  have better contrast, greater sensitivity and better expression levels \citep{hires_gcamp5_2011}. Moreover, other labs are already racing to develop true voltage sensing proteins that can be deployed in eukaryotes \citep{kralj_electrical_2011}. This would allow researchers to visualize neural activity directly instead of inferring activity through calcium transients. 

	Optical neurophysiology is on its way to becoming a mainstay of neuroscience and will likely remain so for the foreseeable future. 
As a result, instrumentation of the sort developed here is going to be ever more critical. 




\subsection*{Where to go from here?}
With the tools presented here it is now possible to  create  quantitative datasets linking neural activity to behavior in \textit{C. elegans}.

 The next step will be to use these datasets to  build data-driven mathematical models of neural circuits. The approach taken to study the vertebrate retina offers inspiration. The retina neuroscience community utilized decades of experimental data to  build  testable quantitative  models of microcircuits within the retina that account for individual computations, like motion detection and background motion suppression \citep{baccus_retinal_2008, gollisch_eye_2010}. In those cases, it was possible to treat collections of neurons as linear filters with quantifiable  impulse response functions. The models of the individual components could, when combined, accurately predict the larger behavior of the neural circuit. 

The tools, techniques and data presented here are a first step towards towards bringing this data-driven  modeling approach to \textit{C. elegans}.

%The ultimate indication that we will have understand how a collection of neurons work together to receive information from the environment, encode that information, and then process it to generate  behavior will not be complete until 



 





