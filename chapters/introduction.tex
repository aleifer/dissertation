%!TEX root = /Users/andy/Documents/Academics/Dissertation/thesis.tex
\begin{savequote}[75mm] 
To cross the threshold from where we are to where we want to be, major conceptual shifts must take place in how we study the brain. One such shift will be from studying elementary processes---single proteins, single genes, and single cells---to studying systems properties---mechanisms made up of many proteins, complex systems of nerve cells, the functioning of whole organisms, and the interaction of groups of organisms. Cellular and molecular approaches will certainly continue to yield important information in the future, but they cannot by themselves unravel the secrets of internal representations of neural circuits or the interaction of circuits---the key steps linking cellular and molecular neuroscience to cognitive neuroscience. 
\qauthor{Eric R. Kandel, \citep{kandel_search_2007}} 
\end{savequote}
\chapter{Introduction}
Neurons are the fundamental unit of the brain.  They are responsible for encoding information from an organism's environment and performing computations to transform that information into purposeful behavior.  The field of systems neuroscience seeks to understand how a network of neurons encodes sensory information and drives behavior on the level of an entire neural circuit or even an entire organism.

Our ability to understand networks of neurons has been driven, in part, by technical developments that allow us to probe the nervous system. 

In this dissertation, I present work that harnesses advances in computer vision, optics and the emerging field of optogenetics to study the neural basis of behavior in the nematode \emph{C. elegans}, an invertebrate model organism that is of itself another powerful tool of the neuroscientist. I put these tools to work studying n	 

\section{Background}
\newthought{How do a collection of neurons} work together to receive information from the environment, encode that information, and then process it to generate  purposeful behavior? In the past this question has been approached different ways and at different scales depending upon the tools at hand. 


In the hands of the brilliant neuroantaomist Cajal, this was a question  of morpohology
Cajal,   with his brilliant eye for anatomy could only see functional connections.. which was informative, and allowed him to articulate hypothesis [see photos of cajal's drawings with arrows]. 

%http://books.google.com/ebooks/reader?id=nysaAAAAYAAJ&printsec=frontcover&output=reader&pg=GBS.PA25


Circuit level analysis.
The development of the multielectrode array \citep{meister_synchronous_1991} \citep{litke_retinal_1991}  led to a 

Later,  generations of electrophysiologists. 
Kandel neurocircuits in applysia.


\subsection{Systems Neuroscience}

Systems neuroscience posits that the appropriate level for analysis is the entire circuit. 
Neuronal basis of behavior. 

\subsection{\textit{C. elegans} as a model organism}


\subsection{Emerging technologies}
Proin entire organism. 

\subsubsection{Optogenetics}
\subsubsection{Computer Vision}



