%!TEX root = /Users/andy/Documents/Academics/Dissertation/thesis.tex
\begin{savequote}[75mm] 
To cross the threshold from where we are to where we want to be, major conceptual shifts must take place in how we study the brain. One such shift will be from studying elementary processes---single proteins, single genes, and single cells---to studying systems properties---mechanisms made up of many proteins, complex systems of nerve cells, the functioning of whole organisms, and the interaction of groups of organisms. Cellular and molecular approaches will certainly continue to yield important information in the future, but they cannot by themselves unravel the secrets of internal representations of neural circuits or the interaction of circuits---the key steps linking cellular and molecular neuroscience to cognitive neuroscience. 
\qauthor{Eric R. Kandel, \citep{kandel_search_2007}} 
\end{savequote}



\chapter{Introduction}

\newthought{How do a collection of neurons} work together to receive information from the environment, encode that information, and then process it to generate  purposeful behavior?  This question motivated my thesis work and prompted me to develop new tools and techniques that use image processing, microscopy and optogenetics to probe the nervous system of the nematode \emph{C. elegans}. That work is covered in Chapters \ref{chapter:colbert}, \ref{chapter:proprioceptive} and  \ref{chapter:omegaTurn}. Along the way, I employed similar techniques  towards the development of  novel fluorescent reporter constructs for next generation microarray and lab-on a chip technologies. That work is discussed in Chapter \ref{chapter:DNAbarcode} and \ref{chapter:DNAbarcodeAnalysis}. The neuronal basis of behavior, however, remains the focus of this introduction.

Neurons are the fundamental unit of the brain.  They are responsible for encoding information from an organism's environment and performing computations to transform that information into  behavior.  The field of systems neuroscience seeks to understand this chain of events from environmental input to motor output, and it seeks to do so at the level of an entire neural circuit or even an entire organism. Our understanding has thus far been limited in part by the lack of tools to probe and observe the activity of ensembles of neurons on a whole-organism scale. Recently, a confluence of advances now make this possible.


\section{\textit{C. elegans} as a model organism}
-C. elegans is a natural tool to study systems neuroscience. 
-The nervous system is compact but tractable. 
-Wiring diagram is known.
-Optically transparent.
-Geneticaly malleable
-Well studied. Built around a large scientific community of resources (WormBase, CGC, WormWeb)

Moreover, there is a long history of using model organisms for neuroscience discovery, including invertebrates such as Aplysia for learning, leech and lamprey for locomotion and Drosophila for a variety of applications.

\section{Optogenetics}
Emerging field that has seen explosive growth in the past six years. Builds off of the increadible advances in molecular genetics that allow one to easily clone genes and add them to organisms in a controlled way.

Give a Brief history of otogenetics, by citing the two main paper.s
 

\section{Optics & Microscopy}
-Inexpensive lasers from china
-Feedback loops and motorized stages

\section{Real-Time Computer Vision} 
To track moving worms, 

computer vision
Advent of accessible real-time computer vision technologyk like OpenCV, generated in part by DARPA challenge http://ai.stanford.edu/~dstavens/

also rely on signal processing and statistical techniques, alhtough in this thesis I delve into those most comprehensively ont eh chapter on DNA barcodes.






in the nervous system with single neuron resolution in awake unrestrained animlas. 

A major goal of this thesis is layhing ouj methods to perturb and record 








 of networks of of neurons has been driven, in part, by technical developments that allow us to probe the nervous system in new ways. 

The Golgi stain  enabled Ramon Y Cajal to first visualize the morphology of neurons and their connections at the turn of the 19th century. During the 20th century electrophysiology enabled scientists to record from stimulate and later patch individual neurons. In the latter part of the 20th century, 







Neurons are the fundamental unit of the brain.  They are responsible for encoding information from an organism's environment and performing computations to transform that information into purposeful behavior.  The field of systems neuroscience seeks to understand how a network of neurons encodes sensory information and drives behavior on the level of an entire neural circuit or even an entire organism.


In this dissertation, I present work that harnesses advances in computer vision, optics and the emerging field of optogenetics to study the neural basis of behavior in the nematode \emph{C. elegans}, an invertebrate model organism that is of itself another powerful tool of the neuroscientist. I put these tools to work studying n	 

 In the past this question has been approached different ways and at different scales depending upon the tools at hand. 


In the hands of the brilliant neuroanatomist Cajal, this was a question of morphology. Cajal prefected the use of the recently invented Golgi stain to label entire neurons 


%http://books.google.com/ebooks/reader?id=nysaAAAAYAAJ&printsec=frontcover&output=reader&pg=GBS.PA25


Circuit level analysis.
The development of the multielectrode array \citep{meister_synchronous_1991} \citep{litke_retinal_1991}  led to a 

Later,  generations of electrophysiologists. 
Kandel neurocircuits in applysia.



Systems neuroscience posits that the appropriate level for analysis is the entire circuit. 
Neuronal basis of behavior. 



\subsection{Emerging technologies}
Proin entire organism. 

\subsubsection{Optogenetics}




