%!TEX root = /Users/andy/Documents/Academics/Dissertation/thesis.tex
\begin{savequote}[75mm] 
To cross the threshold from where we are to where we want to be, major conceptual shifts must take place in how we study the brain. One such shift will be from studying elementary processes---single proteins, single genes, and single cells---to studying systems properties---mechanisms made up of many proteins, complex systems of nerve cells, the functioning of whole organisms, and the interaction of groups of organisms. Cellular and molecular approaches will certainly continue to yield important information in the future, but they cannot by themselves unravel the secrets of internal representations of neural circuits or the interaction of circuits---the key steps linking cellular and molecular neuroscience to cognitive neuroscience. 
\qauthor{Eric R. Kandel, \citep{kandel_search_2007}} 
\end{savequote}



\chapter{Introduction}

\newthought{How do a collection of neurons} work together to receive information from the environment, encode that information, and then process it to generate  purposeful behavior?  This question motivated my thesis work and prompted me to develop new tools and techniques that use image processing, microscopy and optogenetics to probe the nervous system of the nematode \emph{C. elegans}. That work is covered in Chapters \ref{chapter:colbert}, \ref{chapter:proprioceptive} and  \ref{chapter:omegaTurn}. Along the way, I employed similar techniques  towards the development of  novel fluorescent reporter constructs for next generation microarray and lab-on a chip technologies. That work is discussed in Chapter \ref{chapter:DNAbarcode} and \ref{chapter:DNAtheory}. The neuronal basis of behavior, however, remains the focus of this introduction.

\section{Background}
Neurons are the fundamental unit of the brain.  They are responsible for encoding information from an organism's environment and performing computations to transform that information into  behavior.  The field of systems neuroscience seeks to understand this chain of events from environmental input to motor output, and it seeks to do so at the level of an entire neural circuit or even an entire organism. Our understanding has been limited in part by the lack of tools to probe and observe the activity of ensembles of neurons on a whole-organism scale. Recently, a confluence of advances now make this possible.


\subsection{\textit{C. elegans} as a model organism}
The nematode \textit{Caenorhabditis elegans} has emerged as a popular and robust model organism and a natural tool to study systems neuroscience. With only 302 neurons, its nervous system is compact, but tractable. The 1 mm long nematode exhibits a rich array of behaviors \citep{croll_components_1975}. It senses its environment,  navigates towards  food \cite{grewal_migration_1992}  and temperatures \cite{ryu_thermotaxis_2002} that it prefers,  avoids chemicals that it dislikes \citep{croll_behavoural_1975}, and responds to touch \citep{chalfie_neural_1985}. The worm can even learn to associate odors with food that makes it sick \citep{zhang_pathogenic_2005}. 

Starting in the 1970's White et al \citep{white_structure_1976, white_structure_1986}  mapped out the entire wiring diagram of the \textit{C. elegans} nervous system. The individual neurons are morphologically identifiable and their connectivity is stereotyped from one worm to the next. The  scientific community has thus been able to systematically study individual neurons and  used tools such as laser ablation and transgenics to identify which neurons are part of which neural circuits. 

Critically, \textit{C. elegans} is also a major platform for molecular genetics and genetic engineering. It was the first organism into which the fluorescent reporter gene GFP was cloned \citep{chalfie_green_1994}  and it was the first multicellular eukaryote to have its genome sequenced \citep{sulston_c._1992,_genome_1998}. The nematode is especially convenient for genetic engineering. The worms have a fast generation time-- a worm grows from egg to egg-laying adult at room temperature in about four days,  and they are naturally hermaphroditic and self reproduce  making it trivial to maintain isogenic lines. New genes can be added to the worm by injecting plasmids which are incorporated into an extrachromosomal array. The worms are then irradiated, and their DNA damage repair pathway incorporates the plasmids into their chromosomes to form stable transgenic lines.  In the hands of skilled collaborators, the whole process including outcrossing takes less than a month. 

Importantly \textit{C. elegans} is also optically transparent and thus particularly amenable to advances in optical physiology and microscopy, which are the primary tools used in this thesis. 

These attributes have made  \textit{C. elegans}  extremely well studied and as a result, there is a rich repository of knowledge and resources available to the \textit{C. elegans} research community, including:  WormBase, an online database of genes, phenotypes, and publications \citep{harris_wormbase:_2010}; WormAtlas, a comprehensive online anatomical resource providing details of every neuron and cell \citep{altun_wormatlas_2002}; WormBook, a curated collection of review articles and methods \citep{the_c._elegans_research_community_wormbook_2011}; WormWeb, an online interactive network of neural connectivity \citep{bhatla_c._2009};  and the \textit{Caenorhabditis} Genetics Center at the University of Minnesota which acts as a central repository and distributor for transgenic \textit{C. elegans} lines.


There is a long history of using model organisms including invertebrates for neuroscience discovery. In the past \textit{Aplysia} has been used to study learning \citep{castellucci_neuronal_1970}, leech and lamprey for locomotion \citep{briggman_imaging_2006} and \textit{Drosophila} \citep{zhang_drosophila_2010} for a variety of nervous system functions. \textit{C. elegans}, with its compact nervous system, genetic tractability,  optical access and well-mapped neural circuitry is an ideal candidate for studying the neural dynamics underlying behavior.

%OUTLINE NOTES:
%-C. elegans is a natural tool to study systems neuroscience. 
%-The nervous system is compact but tractable. 
%-Wiring diagram is known.
%-Optically transparent.
%-Geneticaly malleable
%-Well studied. Built around a large scientific community of resources (WormBase, CGC, WormWeb)
%-Moreover, there is a long history of using model organisms for neuroscience discovery, including invertebrates such as Aplysia for learning, leech and lamprey for locomotion and Drosophila for a variety of applications.



\subsection{Optical physiology}
Traditional electrophysiology is challenging to perform in the nematode \textit{C. elegans} \citep{goodman_active_1998, schafer_neurophysiological_2006}. The worm's small size and pressurized contents make it difficult for electrodes to gain access to neurons.  As a result, electrophysiology experiments are  performed on worms that are partially dissected and immobilized which negates their ability to elucidate how neural activity correlates to behavior. Advances in optogenetics and fluorescent reporters, however,  offer a viable alternative to electrophysiology and is the technology utilized here.

 

\subsubsection{Optogenetics}
Optogenetics is an emerging field that refers to optical tools based on genetically encoded proteins that manipulate neural function. Karl Deisseroth and Ed Boyden founded the field of optogenetics with their joint development of Channelrhodopsin as an optical method for neural stimulation in  2005 \citep{boyden_millisecond-timescale_2005} and their simultaneous but independent development of Halorhodopsin as a method of neural silencing in 2007 \citep{zhang_multimodal_2007, han_multiple-color_2007}.  For a riveting historical account see \citep{boyden_history_2011}. Optogenetic proteins like Channelrhodopsin and Halorhodopsin are light-activated transmembrane ion channels that open in response to light stimuli at a particular wavelength \citep{nagel_channelrhodopsin-2_2003, yizhar_optogenetics_2011, fenno_development_2011}. These optogenetic proteins were immediately employed  in \textit{C. elegans} and early experiments show how shining light could evoke a touch-like response in the worm  \citep{nagel_light_2005}.  The field has seen explosive growth  just in the past five years. For a review of optogenetics in \textit{C. elegans}, including some of the work presented here see \citep{xu_early_2011} and \citep{yizhar_optogenetics_2011}. In this work I use optogenetics as a tool to probe neural activity in a freely moving worm. 

\subsubsection{Fluorescent Reporters of Neural Activity}
Just as optogenetic proteins allow optical stimulation or inhibition of neural activity,  genetically encoded fluorescent reporters allow optical readouts of neural activity. The first class of these reporters were calcium indicators that functioned by altering their fluorescent properties in response to a cell's calcium levels. Often calcium in a neuron is a good proxy for its membrane potential and thus calcium indicators serve as an indirect measurement of neural activity.   The first genetically encoded calcium indicator was cameleon \citep{miyawaki_fluorescent_1997} which was a calmodulin protein modified by the addition of a Forster Resonance Energy Transfer (FRET) pair of fluorophores. As the calcium levels increase, calmodulin contracts pulling the two attached fluorescing proteins closer together which changes their fluorescence properties. The past decade has seen  steady improvement in genetically encoded calcium indicators \citep{miyawaki_innovations_2005,mank_genetically_2008,yamada_quantitative_2011} and in this work I use one of the most recent indicators, GCaMP3 \citep{tian_imaging_2009}. Of course, it would be ideal to observe the membrane potential directly instead of merely observing calcium levels and the development of true genetically encoded voltage indicators for \textit{C. elegans} is just around the corner \citep{kralj_electrical_2011}. 

Both optogenetics and calcium indicators are non-invasive and thus in principal allows the worm to remain intact and unrestrained. Previously the Samuel lab was the first to manually track a worm and observe its calcium transients as it freely moved \citep{clark_temporal_2007}. In this thesis I develop an automated system to track the worm and either apply optogenetic stimuli or monitor calcium transients.



\subsection{Real-Time Computer Vision}
In addition to the development of optogenetics and fluorescent indicators, the ever decreasing cost of computer power, the rise of inexpensive lasers and microelectromechanical systems (MEMS), and the development of powerful open source computer vision libraries have  all conspired to make the time ripe for an all-optical investigation of neural activity underlying behavior. In particular, computer hardware and open source libraries have just recently risen to the task of performing computer vision analysis in real-time. For example, the development of the OpenCV library \citep{bradski_opencv_2000,bradski_learning_2008}, used heavily in this thesis, was driven in part by the 2004-2007 DARPA grand challenge competitions where teams competed to create a self-driving autonomous vehicles \citep{stavens_learning_2011,buehler_stanley:_2007}.



\subsection{Applications to Behavioral Neuroscience}

By employing the technologies discussed above, this thesis brings us  closer to being able to systematically perturb or monitor neural activity across an entire organism while it is behaving. In this work I employ develop tools explore  neural activity underlying \emph{C. elegans} locomotion. The worm has XXXX muscles, 302 neurons, that have to work together in a coordinated and coherent fashion to undergo locomotion. 



There is immediately questions bout how this takes place. 



\section{Outline}
\subsection{Manipulating Neural Activity in a moving worm}

\subsection{C. elegans Locomotion}
In chapter BLAH we demonstrate how these technqies are combined 


in the nervous system with single neuron resolution in awake unrestrained animlas. 

A major goal of this thesis is layhing ouj methods to perturb and record 












 In the past this question has been approached different ways and at different scales depending upon the tools at hand. 

%http://books.google.com/ebooks/reader?id=nysaAAAAYAAJ&printsec=frontcover&output=reader&pg=GBS.PA25

Circuit level analysis.
The development of the multielectrode array \citep{meister_synchronous_1991} \citep{litke_retinal_1991}  led to a 






