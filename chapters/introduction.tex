\begin{savequote}[75mm] 
To cross the threshold from where we are to where we want to be, major conceptual shifts must take place in how we study the brain. One such shift will be from studying elementary processes---single proteins, single genes, and single cells---to studying systems properties---mechanisms made up of many proteins, complex systems of nerve cells, the functioning of whole organisms, and the interaction of groups of organisms. Cellular and molecular approaches will certainly continue to yield important information in the future, but they cannot by themsleves unravel the secrets of internal representations of neural circuits or the interaction of circuits---the key steps linking cellular and molecular neuroscience to cognitive neuroscience. 
\qauthor{Eric R. Kandel, \citep{kandel_search_2007}} 
\end{savequote}

\chapter{Background}

\newthought{How do a collection of neurons} work together to recieve information from the environment, encode that information, and then process it to generate  purposeful behavior? In the past this question has been approached different ways and at different scales depending upon the tools at hand. 


In the hands of the brilliant neuroantaomist Cajal, this was a question  of morpohology
Cajal,   with his brilliant eye for anatomy could only see functional connections.. which was informative, and allowed him to articulate hypothesis [see photos of cajal's drawings with arrows]. 

%http://books.google.com/ebooks/reader?id=nysaAAAAYAAJ&printsec=frontcover&output=reader&pg=GBS.PA25


Circuit level analysis.
The development of the multielectrode array \citep{meister_synchronous_1991} \citep{litke_retinal_1991}  led to a 

Later,  generations of electrophysiologists. 
Kandel neurocircuits in applysia.


\subsection{Systems Neuroscience}

Systems neuroscience posits that the appropriate level for analysis is the entire circuit. Lorem ipsum dolor sit amet, consectetuer adipiscing elit. Morbi commodo, ipsum sed pharetra gravida, orci magna rhoncus neque, id pulvinar odio lorem non turpis. Nullam sit amet enim. Suspendisse id velit vitae ligula volutpat condimentum. Aliquam erat volutpat. Sed quis velit. Nulla facilisi. Nulla libero. Vivamus pharetra posuere sapien. Nam consectetuer. Sed aliquam, nunc eget euismod ullamcorper, lectus nunc ullamcorper orci, fermentum bibendum enim nibh eget ipsum. Donec porttitor ligula eu dolor. Maecenas vitae nulla consequat libero cursus venenatis. Nam magna enim, accumsan eu, blandit sed, blandit a, eros.

Neuronal basis of behavior. 

\subsection{\textit{C. elegans} as a model organism}


\subsection{Emerging technologies}
Proin entire organism. 

\subsubsection{Optogenetics}
\subsubsection{Computer Vision}



