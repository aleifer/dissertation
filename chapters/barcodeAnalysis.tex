%!TEX root = /Users/andy/Documents/Academics/Dissertation/thesis.tex
\begin{savequote}[75mm] 
This is some random quote to start off the chapter.
\qauthor{Firstname lastname} 
\end{savequote}

\chapter{Maximum Likelihood Estimator Bayesian Inference DNA Origami Barcode Detection}

\newthought{Lorem ipsum dolor sit amet}, consectetuer adipiscing elit. Morbi commodo, ipsum sed pharetra gravida, orci magna rhoncus neque, id pulvinar odio lorem non turpis. Nullam sit amet enim. Suspendisse id velit vitae ligula volutpat condimentum. Aliquam erat volutpat. Sed quis velit. Nulla facilisi. Nulla libero. Vivamus pharetra posuere sapien. Nam consectetuer. Sed aliquam, nunc eget euismod ullamcorper, lectus nunc ullamcorper orci, fermentum bibendum enim nibh eget ipsum. Donec porttitor ligula eu dolor. Maecenas vitae nulla consequat libero cursus venenatis. Nam magna enim, accumsan eu, blandit sed, blandit a, eros.



\begin{equation}
p_{m_j}(f)=p(f-m_j) = \frac{1}{  \sqrt{ (2\pi)^H \det || R_{gh}||} } \exp\left\{ -\frac{1}{2}  (f-m_j)^T R_{nn}^{-1} (f-m_j) \right\}
\end{equation}

Where $R_{nn}$ is the covariance matrix of the noise.

%bold non-ital uppercase for matrices
%bold non-ital lowercase for vectors


%Walk through

Aggregating mean-subtracted ROI's from across the image and finding the variance.

Now we go through the barcodes and try to identify them.

Given a barcode, we find the local background and scale the reference accordingly. 

ASSUMPTION: The camera is just a photon counter. The noise we see comes overwhelmingly from "background fluorescence". We assume it is additive, thus we assume that our signal is merely superposed on top of the mean background. 

This implies that there is a single true peak height that is valid for all peaks. To test this assumption, I should background subtract the local mean from the barcodes and measure the height single isolated peaks. The height of locally background subtracted single isolated peaks should be flat and not correlate with location in the image.

This would mean that when identifying our barcode we don't have to scale our reference, we can use a single reference for all barcodes regardless of the location.


\section{Manuscript Information}
\subsection{Submitted for Publication As}
A version of this chapter has been submitted for publication in the journal \textit{Bioinformatics}.
% in \citep{leifer_optogenetic_2011}:
%\bibentry{leifer_optogenetic_2011}

\subsection{The Author's Contribution}
Andrew M.~Leifer wrote the majority of the manuscript, write all of the software, generated all figures and developed portions of the mathematical framework. Mark C.~Leifer wrote selected passages of the manuscript and developed portions of the mathematical framework. Chenxiang Lin created the DNA origami barcodes and conducted all microscopy work. 
 