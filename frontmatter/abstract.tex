%The abstract! (350 word limit)

We have developed techniques using optogenetics and real-time computer vision to non-invasively manipulate and monitor neural activity with high spatiotemporal resolution in awake behaving \textit{Caenorhabditis elegans}. These methods are used to dissect the nematode's mechanosensory and motor circuits and to elucidate the neural control of wave propagation during forward locomotion.  Additionally, we have used similar computer vision methods to automatically detect and decode fluorescing DNA origami nanobarcodes, a new class of fluorescent reporter constructs.

We first develop an optogenetic illumination system capable of real-time light delivery with high spatial resolution to specified targets in freely moving \textit{C. elegans}, the first of its kind. We use this instrument to probe the nematode's mechanosensory circuit and show that stimulation of a single mechanosensory neuron suffices to induce reversals. We also probe the motor circuit and demonstrate that inhibition of regions of cholinergic motor neurons blocks undulatory wave propagation and that muscle contractions can persist even without inputs from the cholinergic motor neurons.

Next we further probe the motor circuit using optogenetic and microfluidic manipulation. We demonstrate that undulatory wave propagation during forward locomotion depends on stretch-sensitive signaling mediated by cholinergic motor neurons. Specifically we show that posterior body segments are compelled to bend in the same
direction as anterior segments through stretch-sensitive feedback transduced by cholinergic motor neurons.
 
We also develop a tracking system to record intracellular calcium transients in single neurons while simultaneously monitoring macroscopic behavior of freely moving \textit{C. elegans}. We use this system to study the neural activity underlying the worm's stereotyped reversal behavior, called the omega turn. We observe calcium transients in the interneurons AVA, AVB and RIM that align to temporal features of the omega turn.

The experiments so far all rely on optics and computer vision techniques. We apply similar methods to detect DNA origami nanorod barcodes. We develop an optimal Bayesian multiple hypothesis test to unambiguously classify each barcode as a member of one of 216 distinct barcode species.


Overall, this set of experiments demonstrates the powerful role that optogenetics and computer vision can play in behavioral neuroscience and quantitative biophysics.